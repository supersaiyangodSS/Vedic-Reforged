% !TeX program = XeLaTeX
% !TeX root = ../vedamantrabook.tex

\chapt{पवमानसूक्तम्}
\vspace{-1ex}
\centerline{\scriptsize(तैत्तिरीय ब्राह्मणम् अष्टकम् -- १/प्रश्नः -- ४/अनुवाकः -- ८)}\mbox{}\\[-2em]
\centerline{\scriptsize(तैत्तिरीय संहिता काण्डम् -- ५/प्रपाठकः -- ६/अनुवाकः -- १)}


ॐ तच्छं॒ योरावृ॑णीमहे। गा॒तुं य॒ज्ञाय॑। गा॒तुं य॒ज्ञप॑तये। दैवीः᳚ स्व॒स्तिर॑स्तु नः।
स्व॒स्तिर्मानु॑षेभ्यः। ऊ॒र्ध्वं जि॑गातु भेष॒जम्। शं नो॑ अस्तु द्वि॒पदे᳚। शं चतु॑ष्पदे॥
ॐ शान्तिः॒ शान्तिः॒ शान्तिः॑॥

द॒धि॒क्राव्ण्णो॑ अकारिषम्। जि॒ष्णोरश्व॑स्य वा॒जिनः॑।
सु॒र॒भिनो॒ मुखा॑करत्। प्रण॒ आयूꣳ॑षि तारिषत्।

आपो॒ हि ष्ठा म॑यो॒ भुव॒स्ता न॑ ऊ॒र्जे द॑धातन।
म॒हेरणा॑य॒ चक्ष॑से। यो वः॑ शि॒वत॑मो॒ रस॒स्तस्य॑ भाजयते॒ह नः॑।
उ॒श॒तीरि॑व मा॒तरः॑। तस्मा॒ अरं गमाम वो॒ यस्य॒ क्षया॑य॒ जिन्व॑थ।
आपो॑ ज॒नय॑था च नः॥

हिर॑ण्यवर्णाः॒ शुच॑यः पाव॒का यासु॑ जा॒तः क॒श्यपो॒ यास्विन्द्रः॑।
अ॒ग्निं या गर्भं॑ दधि॒रे विरू॑पा॒स्ता न॒ आपः॒ शꣴ स्यो॒ना भ॑वन्तु॥

यासा॒ꣳ॒ राजा॒ वरु॑णो॒ याति॒ मध्ये॑ सत्यानृ॒ते अ॑व॒पश्यं॒ जना॑नाम्।
म॒धु॒श्चुतः॒ शुच॑यो॒ याः पा॑व॒कास्ता न॒ आपः॒ शꣴ स्यो॒ना भ॑वन्तु॥

यासां᳚ दे॒वा दि॒वि कृ॒ण्वन्ति॑ भ॒क्षं या अ॒न्तरि॑क्षे बहु॒धा भव॑न्ति।
याः पृ॑थि॒वीं पय॑सो॒न्दन्ति॑ शु॒क्रास्ता न॒ आपः॒ शꣴ स्यो॒ना भ॑वन्तु॥

शि॒वेन॑ मा॒ चक्षु॑षा पश्यताऽऽपः शि॒वया॑ त॒नुवोप॑ स्पृशत॒ त्वचं॑ मे।
सर्वाꣳ॑ अ॒ग्नीꣳ र॑फ्सु॒षदो॑ हुवे वो॒ मयि॒ वर्चो॒ बल॒मोजो॒ नि ध॑त्त॥

पव॑मानः॒ सुव॒र्जनः॑। प॒वित्रे॑ण॒ विच॑र्‌षणिः। यः पोता॒ स पु॑नातु मा। पु॒नन्तु॑ मा देवज॒नाः।
पु॒नन्तु॒ मन॑वो धि॒या। पु॒नन्तु॒ विश्व॑ आ॒यवः॑। जात॑वेदः प॒वित्र॑वत्। प॒वित्रे॑ण पुनाहि मा।
शु॒क्रेण॑ देव॒दीद्य॑त्। अग्ने॒ क्रत्वा॒ क्रतू॒ꣳ॒ रनु॑। यत्ते॑ प॒वित्र॑म॒र्चिषि॑। अग्ने॒ वित॑तमन्त॒रा।
ब्रह्म॒ तेन॑ पुनीमहे। उ॒भाभ्यां᳚ देवसवितः। प॒वित्रे॑ण स॒वेन॑ च। इ॒दं ब्रह्म॑ पुनीमहे।
वै॒श्व॒दे॒वी पु॑न॒ती दे॒व्यागा᳚त्। यस्यै॑ ब॒ह्वीस्त॒नुवो॑ वी॒तपृ॑ष्ठाः।
तया॒ मद॑न्तः सध॒माद्ये॑षु। व॒यꣴ स्या॑म॒ पत॑यो रयी॒णाम्।
वै॒श्वा॒न॒रो र॒श्मिभि॑र्मा पुनातु। वातः॑ प्रा॒णेने॑षि॒रो म॑यो॒ भूः।
द्यावा॑पृथि॒वी पय॑सा॒ पयो॑भिः। ऋ॒ताव॑री य॒ज्ञिये॑ मा पुनीताम्।
बृ॒हद्भिः॑ सवित॒स्तृभिः॑। वर्{}षि॑ष्ठैर्देव॒मन्म॑भिः।
अग्ने॒ दक्षैः᳚ पुनाहि मा। येन॑ दे॒वा अपु॑नत।
येनाऽऽपो॑ दि॒व्यं कशः॑। तेन॑ दि॒व्येन॒ ब्रह्म॑णा। इ॒दं ब्रह्म॑ पुनीमहे। यः पा॑वमा॒नीर॒ध्येति॑।
ऋषि॑भिः॒ सम्भृ॑त॒ꣳ॒ रसम्᳚। सर्व॒ꣳ॒ स पू॒तम॑श्ञाति।
स्व॒दि॒तं मा॑त॒रिश्व॑ना। पा॒व॒मा॒नीर्यो अ॒ध्येति॑।
ऋषि॑भिः॒ सम्भृ॑त॒ꣳ॒ रसम्᳚। तस्मै॒ सर॑स्वती दुहे। क्षी॒रꣳ स॒र्पिर्मधू॑द॒कम्॥
पा॒व॒मा॒नीः स्व॒स्त्यय॑नीः। सु॒दुघा॒हि पय॑स्वतीः।
ऋषि॑भिः॒ सम्भृ॑तो॒ रसः॑। ब्रा॒ह्म॒णेष्व॒मृतꣳ॑ हि॒तम्।
पा॒व॒मा॒नीर्दि॑शन्तु नः। इ॒मं लो॒कमथो॑ अ॒मुम्।
कामा॒न्थ्सम॑र्धयन्तु नः। दे॒वीर्दे॒वैः स॒माभृ॑ताः।
पा॒व॒मा॒नीः स्व॒स्त्यय॑नीः। सु॒दुघा॒हि घृ॑त॒श्चुतः॑।
ऋषि॑भिः॒ सम्भृ॑तो॒ रसः॑। ब्रा॒ह्म॒णेष्व॒मृतꣳ॑ हि॒तम्।
येन॑ दे॒वाः प॒वित्रे॑ण। आ॒त्मानं॑ पु॒नते॒ सदा᳚।
तेन॑ स॒हस्र॑धारेण। पा॒व॒मा॒न्यः पु॑नन्तु मा।
प्रा॒जा॒प॒त्यं प॒वित्रम्᳚। श॒तोद्या॑मꣳ हिर॒ण्मयम्᳚।
तेन॑ ब्रह्म॒ विदो॑ व॒यम्। पू॒तं ब्रह्म॑ पुनीमहे।
इन्द्रः॑ सुनी॒ती स॒ह मा॑ पुनातु। सोमः॑ स्व॒स्त्या वरु॑णः स॒मीच्या᳚।
य॒मो राजा᳚ प्रमृ॒णाभिः॑ पुनातु मा। जा॒तवे॑दा मो॒र्जय॑न्त्या पुनातु। भूर्भुवः॒ सुवः॑।

तच्छं॒ योरावृ॑णीमहे। गा॒तुं य॒ज्ञाय॑। गा॒तुं य॒ज्ञप॑तये। दैवीः᳚ स्व॒स्तिर॑स्तु नः।
स्व॒स्तिर्मानु॑षेभ्यः। ऊ॒र्ध्वं जि॑गातु भेष॒जम्। शं नो॑ अस्तु द्वि॒पदे᳚। शं चतु॑ष्पदे॥

\centerline{ॐ शान्तिः॒ शान्तिः॒ शान्तिः॑॥}
