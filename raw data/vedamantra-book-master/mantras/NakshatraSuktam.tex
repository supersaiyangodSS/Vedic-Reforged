% !TeX program = XeLaTeX
% !TeX root = ../vedamantrabook.tex

\chapt{नक्षत्रसूक्तम्}
\vspace{-1ex}
\centerline{\scriptsize(तैत्तिरीय ब्राह्मणे अष्टकम् -- ३/प्रश्नः -- १)}\mbox{}\\[-2em]

अ॒ग्निर्नः॑ पातु॒ कृत्ति॑काः।
नक्ष॑त्रं दे॒वमि॑न्द्रि॒यम्।
इ॒दमा॑सां विचक्ष॒णम्।
ह॒विरा॒सं जु॑होतन।
यस्य॒ भान्ति॑ र॒श्मयो॒ यस्य॑ के॒तवः॑।
यस्ये॒मा विश्वा॒ भुव॑नानि॒ सर्वा᳚।
स कृत्ति॑काभि\-र॒भिसं॒वसा॑नः।
अ॒ग्निर्नो॑ दे॒वः सु॑वि॒ते द॑धातु॥१॥ 

प्र॒जाप॑ते रोहि॒णी वे॑तु॒ पत्नी᳚।
वि॒श्वरू॑पा बृह॒ती चि॒त्रभा॑नुः।
सा नो॑ य॒ज्ञस्य॑ सुवि॒ते द॑धातु।
यथा॒ जीवे॑म श॒रदः॒ सवी॑राः।
रो॒हि॒णी दे॒व्युद॑गात्पु॒रस्ता᳚त्।
विश्वा॑ रू॒पाणि॑ प्रति॒मोद॑माना।
प्र॒जाप॑तिꣳ ह॒विषा॑ व॒र्धय॑न्ती।
प्रि॒या दे॒वाना॒मुप॑यातु य॒ज्ञम्॥२॥ 

सोमो॒ राजा॑ मृगशी॒र्‌षेण॒ आगन्।
शि॒वं नक्ष॑त्रं प्रि॒यम॑स्य॒ धाम॑।
आ॒प्याय॑मानो बहु॒धा जने॑षु।
रेतः॑ प्र॒जां यज॑माने दधातु।
यत्ते॒ नक्ष॑त्रं मृगशी॒र्‌षमस्ति॑।
प्रि॒यꣳ रा॑जन् प्रि॒यत॑मं प्रि॒याणा᳚म्।
तस्मै॑ ते सोम ह॒विषा॑ विधेम।
शं न॑ एधि द्वि॒पदे॒ शं चतु॑ष्पदे॥३॥ 

आ॒र्द्रया॑ रु॒द्रः प्रथ॑मा न एति।
श्रेष्ठो॑ दे॒वानां॒ पति॑रघ्नि॒याना᳚म्।
नक्ष॑त्रमस्य ह॒विषा॑ विधेम।
मा नः॑ प्र॒जाꣳ री॑रिष॒न्मोत वी॒रान्।
हे॒ती रु॒द्रस्य॒ परि॑ णो वृणक्तु।
आ॒र्द्रा नक्ष॑त्रं जुषताꣳ ह॒विर्नः॑।
प्र॒मु॒ञ्चमा॑नौ दुरि॒तानि॒ विश्वा᳚।
अपा॒घशꣳ॑ सन्नुदता॒मरा॑तिम्॥४॥ 

पुन॑र्नो दे॒व्यदि॑तिः स्पृणोतु।
पुन॑र्वसू नः॒ पुन॒रेतां᳚ य॒ज्ञम्।
पुन॑र्नो दे॒वा अ॒भिय॑न्तु॒ सर्वे᳚।
पुनः॑ पुनर्वो ह॒विषा॑ यजामः।
ए॒वा न दे॒व्यदि॑तिरन॒र्वा।
विश्व॑स्य भ॒र्त्री जग॑तः प्रति॒ष्ठा।
पुन॑र्वसू ह॒विषा॑ व॒र्धय॑न्ती।
प्रि॒यं दे॒वाना॒मप्ये॑तु॒ पाथः॑॥५॥ 

बृह॒स्पतिः॑ प्रथ॒मं जाय॑मानः।
ति॒ष्यं॑ नक्ष॑त्रम॒भि सम्ब॑भूव।
श्रेष्ठो॑ दे॒वानां॒ पृत॑नासु जि॒ष्णुः।
दि॒शोऽनु॒ सर्वा॒ अभ॑यं नो अस्तु।
ति॒ष्यः॑ पु॒रस्ता॑दु॒त म॑ध्य॒तो नः॑।
बृह॒स्पति॑र्नः॒ परि॑ पातु प॒श्चात्।
बाधे॑तां॒ द्वेषो॒ अभ॑यं कृणुताम्।
सु॒वीर्य॑स्य॒ पत॑यः स्याम॥६॥ 

इ॒दꣳ स॒र्पेभ्यो॑ ह॒विर॑स्तु॒ जुष्टम्᳚।
आ॒श्रे॒षा येषा॑मनु॒यन्ति॒ चेतः॑।
ये अ॒न्तरि॑क्षं पृथि॒वीं क्षि॒यन्ति॑।
ते नः॑ स॒र्पासो॒ हव॒माग॑मिष्ठाः।
ये रो॑च॒ने सूर्य॒स्यापि॑ स॒र्पाः।
ये दिवं॑ दे॒वीमनु॑ स॒ञ्चर॑न्ति।
येषा॑माश्रे॒षा अ॑नु॒यन्ति॒ कामम्᳚।
तेभ्यः॑ स॒र्पेभ्यो॒ मधु॑मज्जुहोमि॥७॥ 

उप॑हूताः पि॒तरो॒ ये म॒घासु॑।
मनो॑जवसः सु॒कृतः॑ सुकृ॒त्याः।
ते नो॒ नक्ष॑त्रे॒ हव॒माग॑मिष्ठाः।
स्व॒धाभि॑र्य॒ज्ञं प्रय॑तं जुषन्ताम्।
ये अ॑ग्निद॒ग्धा येऽन॑ग्निदग्धाः।
ये॑ऽमुं लो॒कं पि॒तरः॑ क्षि॒यन्ति॑।
याꣴश्च॑ वि॒द्म याꣳ उ॑ च॒ न प्र॑वि॒द्म।
म॒घासु॑ य॒ज्ञꣳ सुकृ॑तं जुषन्ताम्॥८॥ 

गवां॒ पतिः॒ फल्गु॑नीनामसि॒ त्वम्।
तद॑र्यमन्वरुणमित्र॒ चारु॑।
तं त्वा॑ व॒यꣳ स॑नि॒तारꣳ॑ सनी॒नाम्।
जी॒वा जीव॑न्त॒मुप॒ संवि॑शेम।
येने॒मा विश्वा॒ भुव॑नानि॒ सञ्जि॑ता।
यस्य॑ दे॒वा अ॑नु सं॒ यन्ति॒ चेतः॑।
अ॒र्य॒मा राजा॒\-ऽजर॒स्तुवि॑ष्मान्।
फल्गु॑नीनामृष॒भो रो॑रवीति॥९॥ 

श्रेष्ठो॑ दे॒वानां᳚ भगवो भगासि।
तत्त्वा॑ विदुः॒ फल्गु॑नी॒स्तस्य॑ वित्तात्।
अ॒स्मभ्यं॑ क्ष॒त्रम॒जरꣳ॑ सु॒वीर्यम्᳚।
गोम॒दश्व॑व॒दुप॒ सन्नु॑\-दे॒ह।
भगो॑ ह दा॒ता भग॒ इत्प्र॑दा॒ता।
भगो॑ दे॒वीः फल्गु॑नी॒रा वि॑वेश।
भग॒स्येत्तं प्र॑स॒वं ग॑मेम।
यत्र॑ दे॒वैः स॑ध॒मादं॑ मदेम॥१०॥ 

आया॑तु दे॒वः स॑वि॒तोप॑यातु।
हि॒र॒ण्यये॑न सु॒वृता॒ रथे॑न।
वह॒न् हस्तꣳ॑ सु॒भगं॑ विद्म॒नाप॑सम्।
प्र॒यच्छ॑न्तं॒ पपु॑रिं॒ पुण्य॒मच्छ॑।
हस्तः॒ प्रय॑च्छत्व॒मृतं॒ वसी॑यः।
दक्षि॑णेन॒ प्रति॑गृभ्णीम एनत्।
दा॒तार॑म॒द्य स॑वि॒ता वि॑देय।
यो नो॒ हस्ता॑य प्रसु॒वाति॑ य॒ज्ञम्॥११॥ 

त्वष्टा॒ नक्ष॑त्रम॒भ्ये॑ति चि॒त्राम्।
सु॒भꣳ स॑सं युव॒तिꣳ रोच॑मानाम्।
नि॒वे॒शय॑न्न॒\-मृता॒न्मर्त्याꣴ॑श्च।
रू॒पाणि॑ पि॒ꣳ॒शन् भुव॑नानि॒ विश्वा᳚।
तन्न॒स्त्वष्टा॒ तदु॑ चि॒त्रा विच॑ष्टाम्।
तन्नक्ष॑त्रं भूरि॒दा अ॑स्तु॒ मह्यम्᳚।
तन्नः॑ प्र॒जां वी॒रव॑तीꣳ सनोतु।
गोभि॑र्नो॒ अश्वैः॒ सम॑नक्तु य॒ज्ञम्॥१२॥ 

वा॒युर्नक्ष॑त्रम॒भ्ये॑ति॒ निष्ट्या᳚म्।
ति॒ग्मशृ॑ङ्गो वृष॒भो रोरु॑वाणः।
स॒मी॒रय॒न् भुव॑ना मात॒रिश्वा᳚।
अप॒ द्वेषाꣳ॑सि नुदता॒मरा॑तीः।
तन्नो॑ वा॒युस्तदु॒ निष्ट्या॑ शृणोतु।
तन्नक्ष॑त्रं भूरि॒दा अ॑स्तु॒ मह्यम्᳚।
तन्नो॑ दे॒वासो॒ अनु॑जानन्तु॒ कामम्᳚।
यथा॒ तरे॑म दुरि॒तानि॒ विश्वा᳚॥१३॥ 

दू॒रम॒स्मच्छत्र॑वो यन्तु भी॒ताः।
तदि॑न्द्रा॒ग्नी कृ॑णुतां॒ तद्विशा॑खे।
तन्नो॑ दे॒वा अनु॑मदन्तु य॒ज्ञम्।
प॒श्चात् पु॒रस्ता॒दभ॑यं नो अस्तु।
नक्ष॑त्राणा॒मधि॑पत्नी॒ विशा॑खे।
श्रेष्ठा॑विन्द्रा॒ग्नी भुव॑नस्य गो॒पौ।
विषू॑चः॒ शत्रू॑नप॒ बाध॑मानौ।
अप॒ क्षुधं॑ नुदता॒मरा॑तिम्॥१४॥ 

पू॒र्णा प॒श्चादु॒त पू॒र्णा पु॒रस्ता᳚त्।
उन्म॑ध्य॒तः पौ᳚र्णमा॒सी जि॑गाय।
तस्यां᳚ दे॒वा अधि॑ सं॒वस॑न्तः।
उ॒त्त॒मे नाक॑ इ॒ह मा॑दयन्ताम्।
पृ॒थ्वी सु॒वर्चा॑ युव॒तिः स॒जोषाः᳚।
पौ॒र्ण॒मा॒स्युद॑गा॒च्छोभ॑माना।
आ॒प्या॒यय॑न्ती दुरि॒तानि॒ विश्वा᳚।
उ॒रुं दुहां॒ यज॑मानाय य॒ज्ञम्॥१५॥ 

ऋ॒द्ध्यास्म॑ ह॒व्यैर्नम॑सोप॒सद्य॑।
मि॒त्रं दे॒वं मि॑त्र॒धेयं॑ नो अस्तु।
अ॒नू॒रा॒धान् ह॒विषा॑ व॒र्धय॑न्तः।
श॒तं जी॑वेम श॒रदः॒ सवी॑राः।
चि॒त्रं नक्ष॑त्र॒मुद॑गात्पु॒रस्ता᳚त्।
अ॒नू॒रा॒धास॒ इति॒ यद्वद॑न्ति।
तन्मि॒त्र ए॑ति प॒थिभि॑र्देव॒यानैः᳚।
हि॒र॒ण्ययै॒र्वित॑तै\-र॒न्तरि॑क्षे॥१६॥ 

इन्द्रो᳚ ज्ये॒ष्ठामनु॒ नक्ष॑त्रमेति।
यस्मि॑न्वृ॒त्रं वृ॑त्र॒तूर्ये॑ त॒तार॑।
तस्मि॑न्व॒यम॒मृतं॒ दुहा॑नाः।
क्षुधं॑ तरेम॒ दुरि॑तिं॒ दुरि॑ष्टिम्।
पु॒र॒न्द॒राय॑ वृष॒भाय॑ धृ॒ष्णवे᳚।
अषा॑ढाय॒ सह॑मानाय मी॒ढुषे᳚।
इन्द्रा॑य ज्ये॒ष्ठा मधु॑म॒द्दुहा॑ना।
उ॒रुं कृ॑णोतु॒ यज॑मानाय लो॒कम्॥१७॥ 

मूलं॑ प्र॒जां वी॒रव॑तीं विदेय।
परा᳚च्येतु॒ निर्‌ऋ॑तिः परा॒चा।
गोभि॒र्नक्ष॑त्रं प॒शुभिः॒ सम॑क्तम्।
अह॑र्भूया॒द्यज॑मानाय॒ मह्यम्᳚।
अह॑र्नो अ॒द्य सु॑वि॒ते द॑धातु।
मूलं॒ नक्ष॑त्र॒मिति॒ यद्वद॑न्ति।
परा॑चीं वा॒चा निर्‌ऋ॑तिं नुदामि।
शि॒वं प्र॒जायै॑ शि॒वम॑स्तु॒ मह्यम्᳚॥१८॥ 

या दि॒व्या आपः॒ पय॑सा सम्बभू॒वुः।
या अ॒न्तरि॑क्ष उ॒त पार्थि॑वी॒र्याः।
यासा॑मषा॒ढा अ॑नु॒यन्ति॒ कामम्᳚।
ता न॒ आपः॒ शꣴ स्यो॒ना भ॑वन्तु।
याश्च॒ कूप्या॒ याश्च॑ ना॒द्याः᳚ समु॒द्रियाः᳚।
याश्च॑ वैश॒न्तीरु॒त प्रा॑स॒चीर्याः।
यासा॑मषा॒ढा मधु॑ भ॒क्षय॑न्ति।
ता न॒ आपः॒ शꣴ स्यो॒ना भ॑वन्तु॥१९॥ 

तन्नो॒ विश्वे॒ उप॑ शृण्वन्तु दे॒वाः।
तद॑षा॒ढा अ॒भिसंय॑न्तु य॒ज्ञम्।
तन्नक्ष॑त्रं प्रथतां प॒शुभ्यः॑।
कृ॒षिर्वृ॒ष्टिर्यज॑मानाय कल्पताम्।
शु॒भ्राः क॒न्या॑ युव॒तयः॑ सु॒पेश॑सः।
क॒र्म॒कृतः॑ सु॒कृतो॑ वी॒र्या॑वतीः।
विश्वा᳚न् दे॒वान् ह॒विषा॑ व॒र्धय॑न्तीः।
अ॒षा॒ढाः काम॒मुप॑ यान्तु य॒ज्ञम्॥२०॥ 

यस्मि॒न् ब्रह्मा॒\-ऽभ्यज॑य॒थ्सर्व॑मे॒तत्।
अ॒मुं च॑ लो॒कमि॒दमू॑ च॒ सर्वम्᳚।
तन्नो॒ नक्ष॑त्रमभि॒जिद्वि॒जित्य॑।
श्रियं॑ दधा॒त्वहृ॑णीय\-मानम्।
उ॒भौ लो॒कौ ब्रह्म॑णा॒ सञ्जि॑ते॒मौ।
तन्नो॒ नक्ष॑त्रमभि॒जिद्विच॑ष्टाम्।
तस्मि॑न्व॒यं पृत॑नाः॒ सञ्ज॑येम।
तन्नो॑ दे॒वासो॒ अनु॑जानन्तु॒ कामम्᳚॥२१॥ 

शृ॒ण्वन्ति॑ श्रो॒णाम॒मृत॑स्य गो॒पाम्।
पुण्या॑मस्या॒ उप॑शृणोमि॒ वाचम्᳚।
म॒हीं दे॒वीं विष्णु॑पत्नीमजू॒र्याम्।
प्र॒तीची॑मेनाꣳ ह॒विषा॑ यजामः।
त्रे॒धा विष्णु॑रुरुगा॒यो विच॑क्रमे।
म॒हीं दिवं॑ पृथि॒वीम॒न्तरि॑क्षम्।
तच्छ्रो॒णैति॒ श्रव॑ इ॒च्छमा॑ना।
पुण्य॒ꣴ॒ श्लोकं॒ यज॑मानाय कृण्व॒ती॥२२॥ 

अ॒ष्टौ दे॒वा वस॑वः सो॒म्यासः॑।
चत॑स्रो दे॒वीर॒जराः॒ श्रवि॑ष्ठाः।
ते य॒ज्ञं पा᳚न्तु॒ रज॑सः प॒रस्ता᳚त्।
सं॒व॒थ्स॒रीण॑म॒मृतꣴ॑ स्व॒स्ति।
य॒ज्ञं नः॑ पान्तु॒ वस॑वः पु॒रस्ता᳚त्।
द॒क्षि॒ण॒तो॑ऽभिय॑न्तु॒ श्रवि॑ष्ठाः।
पुण्यं॒ नक्ष॑त्रम॒भि संवि॑शाम।
मा नो॒ अरा॑तिर॒घश॒ꣳ॒साऽगन्{}॥२३॥ 

क्ष॒त्रस्य॒ राजा॒ वरु॑णोऽधिरा॒जः।
नक्ष॑त्राणाꣳ श॒तभि॑ष॒ग्वसि॑ष्ठः।
तौ दे॒वेभ्यः॑ कृणुतो दी॒र्घमायुः॑।
श॒तꣳ स॒हस्रा॑ भेष॒जानि॑ धत्तः।
य॒ज्ञं नो॒ राजा॒ वरु॑ण॒ उप॑यातु।
तन्नो॒ विश्वे॑ अ॒भि संय॑न्तु दे॒वाः।
तन्नो॒ नक्ष॑त्रꣳ श॒तभि॑षग्जुषा॒णम्।
दी॒र्घमायुः॒ प्रति॑रद्भेष॒जानि॑॥२४॥ 

अ॒ज एक॑पा॒दुद॑गात्पु॒रस्ता᳚त्।
विश्वा॑ भू॒तानि॑ प्रति॒ मोद॑मानः।
तस्य॑ दे॒वाः प्र॑स॒वं य॑न्ति॒ सर्वे᳚।
प्रो॒ष्ठ॒प॒दासो॑ अ॒मृत॑स्य गो॒पाः।
वि॒भ्राज॑मानः समिधा॒न उ॒ग्रः।
आऽन्तरि॑क्षमरुह॒दग॒न्द्याम्।
तꣳ सूर्यं॑ दे॒वम॒जमेक॑पादम्।
प्रो॒ष्ठ॒प॒दासो॒ अनु॑यन्ति॒ सर्वे᳚॥२५॥ 

अहि॑र्बु॒ध्नियः॒ प्रथ॑मान एति।
श्रेष्ठो॑ दे॒वाना॑मु॒त मानु॑षाणाम्।
तं ब्रा᳚ह्म॒णाः सो॑म॒पाः सो॒म्यासः॑।
प्रो॒ष्ठ॒प॒दासो॑ अ॒भि र॑क्षन्ति॒ सर्वे᳚।
च॒त्वार॒ एक॑म॒भि कर्म॑ दे॒वाः।
प्रो॒ष्ठ॒प॒दास॒ इति॒ यान् वद॑न्ति।
ते बु॒ध्नियं॑ परि॒षद्यꣴ॑ स्तु॒वन्तः॑।
अहिꣳ॑ रक्षन्ति॒ नम॑सोप॒सद्य॑॥२६॥ 

पू॒षा रे॒वत्यन्वे॑ति॒ पन्था᳚म्।
पु॒ष्टि॒पती॑ पशु॒पा वाज॑बस्त्यौ।
इ॒मानि॑ ह॒व्या प्रय॑ता जुषा॒णा।
सु॒गैर्नो॒ यानै॒रुप॑यातां य॒ज्ञम्।
क्षु॒द्रान् प॒शून् र॑क्षतु रे॒वती॑ नः।
गावो॑ नो॒ अश्वा॒ꣳ॒ अन्वे॑तु पू॒षा।
अन्न॒ꣳ॒ रक्ष॑न्तौ बहु॒धा विरू॑पम्।
वाजꣳ॑ सनुतां॒ यज॑मानाय य॒ज्ञम्॥२७॥ 

तद॒श्विना॑वश्व॒युजोप॑याताम्।
शुभ॒ङ्गमि॑ष्ठौ सु॒यमे॑भि॒रश्वैः᳚।
स्वं नक्ष॑त्रꣳ ह॒विषा॒ यज॑न्तौ।
मध्वा॒ सम्पृ॑क्तौ॒ यजु॑षा॒ सम॑क्तौ।
यौ दे॒वानां᳚ भि॒षजौ॑ हव्यवा॒हौ।
विश्व॑स्य दू॒ताव॒मृत॑स्य गो॒पौ।
तौ नक्ष॑त्रं जुजुषा॒णोप॑याताम्।
नमो॒ऽश्विभ्यां᳚ कृणुमोऽश्व॒युग्भ्या᳚म्॥२८॥ 

अप॑ पा॒प्मानं॒ भर॑णीर्भरन्तु।
तद्य॒मो राजा॒ भग॑वा॒न् विच॑ष्टाम्।
लो॒कस्य॒ राजा॑ मह॒तो म॒हान् हि।
सु॒गं नः॒ पन्था॒मभ॑यं कृणोतु।
यस्मि॒न्नक्ष॑त्रे य॒म एति॒ राजा᳚।
यस्मि॑न्नेनम॒भ्यषि॑ञ्चन्त दे॒वाः।
तद॑स्य चि॒त्रꣳ ह॒विषा॑ यजाम।
अप॑ पा॒प्मानं॒ भर॑णीर्भरन्तु॥२९॥ 

नि॒वेश॑नी स॒ङ्गम॑नी॒ वसू॑नां॒ विश्वा॑ रू॒पाणि॒ वसू᳚न्यावे॒शय॑न्ती।
स॒ह॒स्र॒पो॒षꣳ सु॒भगा॒ ररा॑णा॒ सा न॒ आग॒न्वर्च॑सा संविदा॒ना॥ यत्ते॑ दे॒वा अद॑धुर्भाग॒धेय॒ममा॑वास्ये सं॒वस॑न्तो महि॒त्वा।
सा नो॑ य॒ज्ञं पि॑पृहि विश्ववारे र॒यिं नो॑ धेहि सुभगे सु॒वीरम्᳚॥३०॥ 